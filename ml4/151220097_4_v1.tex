\documentclass[a4paper,UTF8]{article}
\usepackage{ctex}
\usepackage[margin=1.25in]{geometry}
\usepackage{color}
\usepackage{graphicx}
\usepackage{amssymb}
\usepackage{amsmath}
\usepackage{amsthm}
\usepackage{enumerate}
\usepackage{bm}
\usepackage{hyperref}
\usepackage{epsfig}
\usepackage{color}
\usepackage{mdframed}
\usepackage{lipsum}
\usepackage{mathtools}
\newmdtheoremenv{thm-box}{myThm}
\newmdtheoremenv{prop-box}{Proposition}
\newmdtheoremenv{def-box}{定义}

\setlength{\evensidemargin}{.25in}
\setlength{\textwidth}{6in}
\setlength{\topmargin}{-0.5in}
\setlength{\topmargin}{-0.5in}
% \setlength{\textheight}{9.5in}
%%%%%%%%%%%%%%%%%%此处用于设置页眉页脚%%%%%%%%%%%%%%%%%%
\usepackage{fancyhdr}                                
\usepackage{lastpage}                                           
\usepackage{layout}                                             
\footskip = 10pt 
\pagestyle{fancy}                    % 设置页眉                 
\lhead{2018年春季}                    
\chead{机器学习导论}                                                
% \rhead{第\thepage/\pageref{LastPage}页} 
\rhead{作业四}                                                                                               
\cfoot{\thepage}                                                
\renewcommand{\headrulewidth}{1pt}  			%页眉线宽,设为0可以去页眉线
\setlength{\skip\footins}{0.5cm}    			%脚注与正文的距离           
\renewcommand{\footrulewidth}{0pt}  			%页脚线宽,设为0可以去页脚线

\makeatletter 									%设置双线页眉                                        
\def\headrule{{\if@fancyplain\let\headrulewidth\plainheadrulewidth\fi%
\hrule\@height 1.0pt \@width\headwidth\vskip1pt	%上面线为1pt粗  
\hrule\@height 0.5pt\@width\headwidth  			%下面0.5pt粗            
\vskip-2\headrulewidth\vskip-1pt}      			%两条线的距离1pt        
 \vspace{6mm}}     								%双线与下面正文之间的垂直间距              
\makeatother  

%%%%%%%%%%%%%%%%%%%%%%%%%%%%%%%%%%%%%%%%%%%%%%
\numberwithin{equation}{section}
%\usepackage[thmmarks, amsmath, thref]{ntheorem}
\newtheorem{myThm}{myThm}
\newtheorem*{myDef}{Definition}
\newtheorem*{mySol}{Solution}
\newtheorem*{myProof}{Proof}
\newcommand{\indep}{\rotatebox[origin=c]{90}{$\models$}}
\newcommand*\diff{\mathop{}\!\mathrm{d}}

\usepackage{multirow}

%--

%--
\begin{document}
\title{机器学习导论\\
作业四}
\author{151220097, 孙旭东, 248381185@qq.com}
\maketitle

\section{[30pts] Kernel Methods}
Mercer定理告诉我们对于一个二元函数$k(\cdot,\cdot)$,它是正定核函数当且仅当对任意$N$和$\mathbf{x}_1, \mathbf{x}_2,\ldots,\mathbf{x}_N$,它对应的核矩阵是半正定的. 假设$k_1(\cdot,\cdot)$和$k_2(\cdot,\cdot)$分别是关于核矩阵$K_1$和$K_2$的正定核函数. 另外,核矩阵$K$中的元素为$K_{ij}=k(\mathbf{x}_i,\mathbf{x}_j)$. 请根据Mercer定理证明对应于以下核矩阵的核函数正定.
\begin{enumerate}[(1)]
	\item \textbf{[10pts]} $K_3=a_1K_1+a_2K_2$, 其中$a_1,a_2\geq 0$.
	\item \textbf{[10pts]} $f(\cdot)$是任意实值函数,由$k_4(\mathbf{x},\mathbf{x}')=f(\mathbf{x})f(\mathbf{x}')$定义的$K_4$.
	\item \textbf{[10pts]} 由$k_5(\mathbf{x},\mathbf{x}')=k_1(\mathbf{x},\mathbf{x}')k_2(\mathbf{x},\mathbf{x}')$定义的$K_5$.
\end{enumerate}

\begin{mySol}
\begin{enumerate}[(1)]
~\\
\item
要证明$K_3$对应的核函数正定,即证明$K_3$半正定。\\
由于$K_1$和$K_2$对应的核函数是正定核函数,所以$K_1$和$K_2$都是半正定矩阵,所以$K_1$和$K_2$是对称的,所以显然$K_3$也是对称的。\\
又因为$K_1$和$K_2$是半正定矩阵,所以还有:对任意非零向量$\mathbf{x}$,都有:\\
\begin{equation}
\mathbf{x}^\mathrm{T}K_1\mathbf{x} \geq 0
\end{equation}
\begin{equation}
\mathbf{x}^\mathrm{T}K_2\mathbf{x} \geq 0
\end{equation}
又因为$a_1,a_2\geq 0$,所以:
\begin{equation}
\mathbf{x}^\mathrm{T}a_1K_1\mathbf{x} \geq 0
\end{equation}
\begin{equation}
\mathbf{x}^\mathrm{T}a_2K_2\mathbf{x} \geq 0
\end{equation}
那么可以得到:
\begin{equation}
\mathbf{x}^\mathrm{T}K_3\mathbf{x} = \mathbf{x}^\mathrm{T}(a_1K_1+a_2K_2)\mathbf{x} = \mathbf{x}^\mathrm{T}a_1K_1\mathbf{x} + \mathbf{x}^\mathrm{T}a_2K_2\mathbf{x} \geq 0
\end{equation}
所以可得$K_3$是半正定矩阵,所以对应的核函数是正定的。
\item 
设$K_{ij}$表示$K_4$的$i$行,第$j$列的项。首先因为$K_{ij}=f(\mathbf{x}_i)f(\mathbf{x}_j)=f(\mathbf{x}_j)f(\mathbf{x}_i)=K_{ji}$,所以$K_4$是对称的。对于任意非零向量$\mathbf{x}$,用$x_i$表示$\mathbf{x}$的第$i$个项,可以有:
\begin{equation}
\mathbf{x}^\mathrm{T}K_4\mathbf{x} = \sum_{i=1}^{N}\sum_{j=1}^{N}x_iK_{ij}x_j
\end{equation}
又因为$k_4(\mathbf{x}_i,\mathbf{x}_j)=f(\mathbf{x}_i)f(\mathbf{x}_j)$,所以为了方便证明令$t_i = f(\mathbf{x}_i)$,可得$K_{ij}=k_4(\mathbf{x}_i,\mathbf{x}_j)=f(\mathbf{x}_i)f(\mathbf{x}_j)=t_it_j$,结合上式可得:
\begin{equation}
\begin{aligned}
\mathbf{x}^\mathrm{T}K_4\mathbf{x} =& \sum_{i=1}^{N}\sum_{j=1}^{N}x_iK_{ij}x_j = \sum_{i=1}^{N}\sum_{j=1}^{N}x_it_it_jx_j\\ 
=& \sum_{i=1}^{N}(x_it_i)^2 + 2\sum_{i=1}^{N-1}\sum_{j=i+1}^{N}x_it_ix_jt_j\\
=& (t_1x_1 + t_2x_2 +...+ t_nx_n)^2 \geq 0
\end{aligned}
\end{equation}
所以可得$K_4$是半正定矩阵,所以对应的核函数正定。
\item 
为了方便证明,令$A=K_1$,$B=K_2$,$C=K_5$。然后再定义$A_{ij}$表示$A$的第$i$行第$j$列的元素,即$A_{ij}=k_1(\mathbf{x}_i, \mathbf{x}_j)$,同理有$B_{ij}=k_2(\mathbf{x}_i, \mathbf{x}_j)$,$C_{ij}=k_5(\mathbf{x}_i, \mathbf{x}_j)$。根据题目定义可得$C_{ij} = A_{ij}B_{ij}$。\\
因为$k_1$和$k_2$都是正定核函数,所以$A$和$B$都是半正定矩阵。所以$A$和$B$都是对称的,所以得到$k_5(\mathbf{x}_i,\mathbf{x}_j)=k_1(\mathbf{x}_i,\mathbf{x}_j)k_2(\mathbf{x}_i,\mathbf{x}_j)=k_1(\mathbf{x}_j,\mathbf{x}_i)k_2(\mathbf{x}_j,\mathbf{x}_i)=k_5(\mathbf{x}_j,\mathbf{x}_i)$,所以$C$是实对称矩阵。又因为$B$是半正定矩阵,所以根据半正定矩阵的性质可知,通过使用Cholesky分解,一定存在N阶实矩阵$L$,满足$B= LL^\mathrm{T}$,使用$L_{ij}$表示$L$的每个元素,根据定义可得$B_{ij} = \sum_{k=1}^{N}L_{ik}L_{jk}$。对于任意非零向量$\mathbf{x}$,用$x_i$表示$\mathbf{x}$的第$i$个项,可以有:
\begin{equation}
\mathbf{x}^\mathrm{T}C\mathbf{x} = \sum_{i=1}^{N}\sum_{j=1}^{N}x_iC_{ij}x_j
\end{equation}
再根据以上定义,有
\begin{equation}
\begin{aligned}
\mathbf{x}^\mathrm{T}C\mathbf{x} =& \sum_{i=1}^{N}\sum_{j=1}^{N}x_iC_{ij}x_j\\
=& \sum_{i=1}^{N}\sum_{j=1}^{N}x_iA_{ij}B_{ij}x_j\\
=& \sum_{i=1}^{N}\sum_{j=1}^{N}x_iA_{ij}x_j\sum_{k=1}^{N}L_{ik}L_{jk}\\
=& \sum_{k=1}^{N}\sum_{i=1}^{N}\sum_{j=1}^{N}(x_iL_{ik})A_{ij}(x_jL_{jk})\\
\end{aligned}
\end{equation}
然后定义向量$\mathbf{y}_k=(x_1L_{1k},x_2L_{2k},\cdots,x_nL_{nk})$,所以有
\begin{equation}
\mathbf{x}^\mathrm{T}C\mathbf{x} = \sum_{k=1}^{N}\mathbf{y}_k^\mathrm{T}A\mathbf{y}_k
\end{equation}
又因为$A$是半正定矩阵,所以有$\mathbf{y}_k^\mathrm{T}A\mathbf{y}_k \geq 0$,所以可得
\begin{equation}
\mathbf{x}^\mathrm{T}C\mathbf{x} = \sum_{k=1}^{N}\mathbf{y}_k^\mathrm{T}A\mathbf{y}_k \geq 0
\end{equation}
所以可得$C$是半正定矩阵,所以$K_5$是半正定矩阵,所以对应的核函数正定。
\end{enumerate}
\end{mySol}
\newpage

\section{[25pts] SVM with Weighted Penalty}
考虑标准的SVM优化问题如下(即课本公式(6.35)),
\begin{equation}
	\label{eq-svm}
	\begin{split}
		\min_{\mathbf{w},b,\xi_i}& \quad \frac{1}{2} \lVert \mathbf{w} \rVert^2 + C\sum_{i=1}^m\xi_i\\
		\text{s.t.}&  \quad y_i(\mathbf{w}^\mathrm{T}\mathbf{x}_i + b)\geq 1-\xi_i\\
		& \quad \xi_i \geq 0, i = 1,2,\cdots,m.
	\end{split}
\end{equation}

注意到,在\eqref{eq-svm}中,对于正例和负例,其在目标函数中分类错误的“惩罚”是相同的. 在实际场景中,很多时候正例和负例错分的“惩罚”代价是不同的. 比如考虑癌症诊断问题,将一个确实患有癌症的人误分类为健康人,以及将健康人误分类为患有癌症,产生的错误影响以及代价不应该认为是等同的.

现在,我们希望对负例分类错误的样本(即false positive)施加$k>0$倍于正例中被分错的样本的“惩罚”. 对于此类场景下,
\begin{enumerate}[(1)]
\item \textbf{[10pts]} 请给出相应的SVM优化问题.
\item \textbf{[15pts]} 请给出相应的对偶问题,要求详细的推导步骤,尤其是如KKT条件等.
\end{enumerate}
\begin{mySol}
	此处用于写解答(中英文均可)
\begin{enumerate}[(1)]
\item 
定义$\mathbf{P}$表示正例的下标的集合,$\mathbf{N}$表示负例的下标的集合,相应的SVM优化问题为:
\begin{equation}
\begin{split}
\min_{\mathbf{w},b,\xi_i}& \quad \frac{1}{2} \lVert \mathbf{w} \rVert^2 + C\sum_{i \in \mathbf{P}}^m\xi_i + Ck\sum_{i \in \mathbf{N}}^m\xi_i\\
\text{s.t.}&  \quad y_i(\mathbf{w}^\mathrm{T}\mathbf{x}_i + b)\geq 1-\xi_i\\
& \quad \xi_i \geq 0, i = 1,2,\cdots,m.
\end{split}
\end{equation}

\item 
根据上式$(2.2)$,使用拉格朗日乘子法可得拉格朗日函数如下:
\begin{equation}
\begin{aligned}
L(\mathbf{w},b,\boldsymbol{\alpha},\boldsymbol{\xi},\boldsymbol{\mu}) =& \frac{1}{2} \lVert \mathbf{w} \rVert^2 + C\sum_{i \in \mathbf{P}}^m\xi_i + Ck\sum_{i \in \mathbf{N}}^m\xi_i\\
& + \sum_{i=1}^{m}\alpha_i(1-\xi_i-y_i(\mathbf{w}^\mathrm{T}\mathbf{x}_i + b)) - \sum_{i=1}^{m}\mu_i\xi_i 
\end{aligned}
\end{equation}
其中$\alpha_i \geq 0$,$\mu_i \geq 0$是拉格朗日乘子。\\
令$L(\mathbf{w},b,\boldsymbol{\alpha},\boldsymbol{\xi},\boldsymbol{\mu})$对$\mathbf{w}$,$b$,$\xi_i$的偏导为零可得
\begin{equation}
	\mathbf{w} = \sum_{i=1}^{m}\alpha_iy_i\mathbf{x}_i
\end{equation}
\begin{equation}
	0=\sum_{i=1}^{m}\alpha_iy_i
\end{equation}
\begin{equation}
	C = (\alpha_i + \mu_i)(\mathbb{I}(i\in \mathbf{P}) + \frac{1}{k}\mathbb{I}(i\in \mathbf{N}) )
\end{equation}
为了方便解答,这里用到了指示器函数,定义如下:
\begin{equation}
\mathbb{I}(\cdot) = 
\begin{cases}
1 & \mbox{$\cdot$ is true}\\
0 & \mbox{$\cdot$ is false}
\end{cases}
\end{equation}
现在将$(2.4)-(2.6)$代入式$(2.3)$即可得到式$(2.2)$的对偶问题
\begin{equation}
\begin{split}
\max_{\alpha_i}& \quad \sum_{i=1}^{m}\alpha_i-\frac{1}{2}\sum_{i=1}^{m}\sum_{j=1}^{m}\alpha_i\alpha_jy_iy_j\mathbf{x}_i^\mathrm{T}\mathbf{x}_j\\
\text{s.t.}&  \quad \sum_{i=1}^{m}\alpha_iy_i = 0\\
& \quad 0 \leq \alpha_i \leq C(\mathbb{I}(i\in \mathbf{P}) + k\mathbb{I}(i\in \mathbf{N}) ), i = 1,2,\cdots,m.
\end{split}	
\end{equation}
KKT条件要求为
\begin{equation}
\begin{cases}
\alpha_i \geq 0\\
\mu_i \geq 0\\
\xi_i \geq 0\\
y_i(\mathbf{w}^\mathrm{T}\mathbf{x}_i + b) - 1 + \xi_i \geq 0\\
\alpha_i(y_i(\mathbf{w}^\mathrm{T}\mathbf{x}_i + b) - 1 + \xi_i) = 0\\
\mu_i\xi_i = 0
\end{cases}
\end{equation}
\end{enumerate}
\end{mySol}
\newpage


\section{[30pts+10*pts] Nearest Neighbor}
假设数据集$\{\mathbf{x}_1,...,\mathbf{x}_n\}$是从一个以$\mathbf{0}$为中心的$p$维单位球中独立均匀采样而得到的$n$个样本点. 这个球可以表示为:
\begin{equation}
	B = \{\mathbf{x}: \lVert \mathbf{x} \rVert^2 \leq 1\} \subset \mathbb{R}^p.
\end{equation}
其中,$\lVert \mathbf{x} \rVert= \sqrt{\langle\mathbf{x}, \mathbf{x}\rangle}$,$\langle\mathbf{x}, \mathbf{x}\rangle$是$\mathbb{R}^p$空间中向量的内积.
在本题中,我们将探究原点$O$与其最近邻($1$-NN)的距离$d^*$,以及这个距离$d^*$与$p$之间的关系. 在这里,我们将原点$O$以及其$1$-NN之间的距离定义为:
\begin{equation}
d^* \coloneqq \min_{1\leq i \leq n} \lVert \mathbf{x}_i \rVert,
\end{equation}
不难发现$d^*$是一个随机变量,因为$\mathbf{x}_i$是随机产生的.
\begin{enumerate}[(1)]
\item \textbf{[5pts]} 当$p=1$且$t\in [0,1]$时,请计算$\Pr(d^*\leq t)$,即随机变量$d^*$的累积分布函数(Cumulative Distribution Function, $\mathbf{CDF}$).
\item \textbf{[10pts]} 请写出$d^*$的$\mathbf{CDF}$的一般公式,即当$p\in\{1,2,3,...\}$时$d^*$对应的取值. 提示:半径为$r$的$p$维球体积是:
	\begin{equation}
	V_p(r)=\frac{(r\sqrt{\pi})^p}{\Gamma(p/2+1)},
	\end{equation}
	其中,$\Gamma(1/2)=\sqrt{\pi}$,$\Gamma(1)=1$,且有$\Gamma(x+1)=x\Gamma(x)$对所有的$x>0$成立;并且对于$n\in\mathbb{N}^*$,有$\Gamma(n+1)=n!$.
\item \textbf{[10pts]} 请求解随机变量$d^*$的中位数,即使得$\Pr(d^*\leq t)=1/2$成立时的$t$值. 答案是与$n$和$p$相关的函数.

\item \textbf{[附加题10pts]} 请通过$\mathbf{CDF}$计算使得原点$O$距其最近邻的距离$d^*$小于$1/2$的概率至少$0.9$的样本数$n$的大小. 提示:答案仅与$p$相关. 你可能会用到$\ln(1-x)$的泰勒展开式:
\begin{equation}
\ln(1-x) = - \sum_{i=1}^{\infty} \frac{x^i}{i}, \quad \mbox{for } -1 \leq x < 1.
\end{equation}
\item \textbf{[5pts]} 在解决了以上问题后,你关于$n$和$p$以及它们对$1$-NN的性能影响有什么理解.
\end{enumerate}
\begin{mySol}
	此处用于写解答(中英文均可)
\begin{enumerate}[(1)]
\item 因为$p=1$,所以可得$d^* \coloneqq \min_{1\leq i \leq n} x_i$。要计算$\Pr(d^*\leq t)$,可以先计算$\Pr(d^* > t)$。定义事件$E_i$表示$x_i > t$,考虑到$d^* \coloneqq \min_{1\leq i \leq n} x_i$,所以$d^* > t$等价于对任意$i=1,2,\cdots,n$,都有$x_i > t$。所以$Pr(d^* > t) = Pr(E_1 \land E_2 \land \cdots \land E_n)$,再考虑到n个样本独立均匀的采样得到的,所以:
\begin{equation}
\begin{aligned}
Pr(d^* > t) =& Pr(E_1 \land E_2 \land \cdots \land E_n)\\
=& Pr(E_1)Pr(E_2)\cdots Pr(E_n)\\
=& Pr(x_1>t)Pr(x_2>t)\cdots Pr(x_n>t)\\
=& (1-t)^n
\end{aligned}
\end{equation}
所以可得
\begin{equation}
Pr(d^* \leq t) = 1 - Pr(d^* > t) = 1 - (1-t)^n
\end{equation}
\item 先考虑$Pr(\lVert \mathbf{x}_i \rVert \leq t)$。根据独立均匀采样,球体的定义以及$\lVert \mathbf{x} \rVert= \sqrt{\langle\mathbf{x}, \mathbf{x}\rangle}$,可得满足$\lVert \mathbf{x}_i \rVert \leq t$的$\mathbf{x}_i$构成了一个半径为t的p维球体,所以可得:
\begin{equation}
Pr(\lVert \mathbf{x}_i \rVert > t) = 1 - Pr(\lVert \mathbf{x}_i \rVert \leq t) = 1 - \frac{V_p(t)}{V_p(1)}
\end{equation}
根据p维球体的定义,可得:
\begin{equation}
\frac{V_p(t)}{V_p(1)} = \frac{t^p}{1}
\end{equation}
结合上一问可得
\begin{equation}
\begin{aligned}
Pr(d^* > t) =& Pr(E_1 \land E_2 \land \cdots \land E_n)\\
=& Pr(E_1)Pr(E_2)\cdots Pr(E_n)\\
=& Pr(\lVert \mathbf{x}_1\rVert > t)Pr(\lVert \mathbf{x}_2\rVert > t)\cdots Pr(\lVert \mathbf{x}_n\rVert > t)\\
=& (1 - t^p)^n
\end{aligned}
\end{equation}
所以可得
\begin{equation}
Pr(d^* \leq t) = 1 - Pr(d^* > t) = 1 - (1-t^p)^n
\end{equation}
\item 
因为$Pr(d^*\leq t)=\frac{1}{2}$,所以有$1 - (1-t^p)^n = \frac{1}{2}$,可得
\begin{equation}
\begin{aligned}
&1 - (1-t^p)^n = 1/2\\
\Leftrightarrow\ &1/2 = (1-t^p)^n\\
\Leftrightarrow\ &\sqrt[n]{1/2} = 1-t^p\\
\Leftrightarrow\ &t^p = 1- \sqrt[n]{1/2}\\
\Leftrightarrow\ &t = \sqrt[p]{1- \sqrt[n]{1/2}}
\end{aligned}
\end{equation}
所以可得$t = \sqrt[p]{1- \sqrt[n]{1/2}}$。
\item 
根据第$(2)$题答案可得$Pr(d^* \leq \frac{1}{2}) = 1 - (1-(1/2)^p)^n$,令$1 - (1-(1/2)^p)^n \geq 0.9$,可得
\begin{equation}
\begin{aligned}
&1 - (1-(1/2)^p)^n \geq 0.9\\
\Leftrightarrow\ &0.1 \geq (1-(1/2)^p)^n\\
\Leftrightarrow\ &n \geq \log_{1-(1/2)^p}0.1\\
\Leftrightarrow\ &n \geq \frac{\ln (1-0.9)}{\ln (1-(1/2)^p)}
\end{aligned}
\end{equation}
\end{enumerate}
\end{mySol}
\newpage

\section{[15pts] Principal Component Analysis}
一些经典的降维方法,例如PCA,可以将均值为$\mathbf{0}$的高维数据通过对其协方差矩阵的特征值计算,取较高特征值对应的特征向量的操作而后转化为维数较低的数据. 在这里,我们记$U_k$为$d\times k$的矩阵,这个矩阵是由原数据协方差矩阵最高的$k$个特征值对应的特征向量组成的. 

在这里我们有两种方法来求出低维的对应于$\mathbf{x}\in \mathbb{R}^d$的重构向量$\mathbf{w}\in \mathbb{R}^k$:

A. 利用$U_k\mathbf{w}$重构出对应的$\mathbf{x}$时, 最小化重构平方误差;

B. 将$\mathbf{x}$投影在由$U_k$的列向量张成的空间中.

在这里,我们将探究这两种方法的关系.
\begin{enumerate}[(1)]
	\item \textbf{[5pts]} 写出方法A中最小化重构平方误差的目标函数的表示形式.
	\item \textbf{[10pts]} 证明通过方法A得到的重构向量就是$U_k^\mathrm{T}\mathbf{x}$,也就是$\mathbf{x}$在$U_k$列向量空间中的投影 (通过方法B得到的重构向量). 这里, 有$U_k^\mathrm{T}U_k=I_k$成立, 其中的$I_k$是$k\times k$的单位矩阵.
\end{enumerate}
\begin{mySol}
	此处用于写解答(中英文均可)
\begin{enumerate}[(1)]
\item 
假设共计有m个样例高维向量$\mathbf{x}_1,\mathbf{x}_2,\cdots,\mathbf{x}_m$,他们对应的低维向量为$\mathbf{w}_1,\mathbf{w}_2,\cdots,\mathbf{w}_m$。所有的$\mathbf{x}_i$的均值为$\mathbf{0}$。用$U_k$和$\mathbf{w}_i$重构出$\hat{\mathbf{x}}_i$,即
\begin{equation}
\hat{\mathbf{x}}_i = U_k\mathbf{w}_i
\end{equation}
为了最小化重构平方误差,可得目标函数
\begin{equation}
\min_{\hat{\mathbf{x}}_i} \quad \sum_{i=1}^{m}\lVert \hat{\mathbf{x}}_i - \mathbf{x}_i\rVert_2^2	
\end{equation}
即
\begin{equation}
\min_{\mathbf{w}_i} \quad \sum_{i=1}^{m}\lVert U_k\mathbf{w}_i - \mathbf{x}_i\rVert_2^2	
\end{equation}
或者写为
\begin{equation}
\min_{\mathbf{w}_i} \quad \sum_{i=1}^{m}(U_k\mathbf{w}_i - \mathbf{x}_i)^\mathrm{T}(U_k\mathbf{w}_i - \mathbf{x}_i)
\end{equation}
\item 
令$f(\mathbf{w}_1,\mathbf{w}_2,\cdots,\mathbf{w}_m)=\sum_{i=1}^{m}\lVert U_k\mathbf{w}_i - \mathbf{x}_i\rVert_2^2$。为了得到$(4.3)$的解,即让$f$对$\mathbf{w}_1,\cdots,\mathbf{w}_m$求偏导,然后令导数为0,即
\begin{equation}
\begin{aligned}
\frac{\partial f}{\partial \mathbf{w}_1} =& 2U_k^\mathrm{T}(U_k\mathbf{w}_1 - \mathbf{x}_1)=0\\
\frac{\partial f}{\partial \mathbf{w}_2} =& 2U_k^\mathrm{T}(U_k\mathbf{w}_2 - \mathbf{x}_2)=0\\
&\cdots\cdots\\
\frac{\partial f}{\partial \mathbf{w}_m} =& 2U_k^\mathrm{T}(U_k\mathbf{w}_m - \mathbf{x}_m)=0\\
\end{aligned}
\end{equation}
可得,对任意的$i=1,2,\cdots,m$,都有
\begin{equation}
\begin{aligned}
&2U_k^\mathrm{T}(U_k\mathbf{w}_i - \mathbf{x}_i)=0\\
\Leftrightarrow\ &U_k^\mathrm{T}U_k\mathbf{w}_i = U_k^\mathrm{T}\mathbf{x}_i\\
\Leftrightarrow\ &I_k\mathbf{w}_i = U_k^\mathrm{T}\mathbf{x}_i\\
\Leftrightarrow\ &\mathbf{w}_i = U_k^\mathrm{T}\mathbf{x}_i
\end{aligned}
\end{equation}
根据以上可得对于$\mathbf{x}_i$,通过方法A得到的重构向量$\mathbf{w}_i$就是$U_k^\mathrm{T}\mathbf{x}_i$。所以得证通过方法A得到的重构向量就是$U_k^\mathrm{T}\mathbf{x}$,也就是$\mathbf{x}$在$U_k$列向量空间中的投影 (通过方法B得到的重构向量)。
\end{enumerate}
\end{mySol}

\end{document}